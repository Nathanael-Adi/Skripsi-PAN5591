%versi 3 (18-12-2016)
\chapter{Kode Program}
\label{lamp:A}

%terdapat 2 cara untuk memasukkan kode program
% 1. menggunakan perintah \lstinputlisting (kode program ditempatkan di folder yang sama dengan file ini)
% 2. menggunakan environment lstlisting (kode program dituliskan di dalam file ini)
% Perhatikan contoh yang diberikan!!
%
% untuk keduanya, ada parameter yang harus diisi:
% - language: bahasa dari kode program (pilihan: Java, C, C++, PHP, Matlab, C#, HTML, R, Python, SQL, dll)
% - caption: nama file dari kode program yang akan ditampilkan di dokumen akhir
%
% Perhatian: Abaikan warning tentang textasteriskcentered!!
%


% \begin{lstlisting}[language=Java, caption=MyCode.c]

% // This does not make algorithmic sense, 
% // but it shows off significant programming characters.

% #include<stdio.h>

% void myFunction( int input, float* output ) {
% 	switch ( array[i] ) {
% 		case 1: // This is silly code
% 			if ( a >= 0 || b <= 3 && c != x )
% 				*output += 0.005 + 20050;
% 			char = 'g';
% 			b = 2^n + ~right_size - leftSize * MAX_SIZE;
% 			c = (--aaa + &daa) / (bbb++ - ccc % 2 );
% 			strcpy(a,"hello $@?"); 
% 	}
% 	count = ~mask | 0x00FF00AA;
% }

% // Fonts for Displaying Program Code in LATEX
% // Adrian P. Robson, nepsweb.co.uk
% // 8 October 2012
% // http://nepsweb.co.uk/docs/progfonts.pdf

% \end{lstlisting}

\begin{lstlisting}[language=HTML, caption=App.tsx]
import { IonApp, IonRouterOutlet, IonSplitPane, setupIonicReact } from '@ionic/react';
import { IonReactRouter } from '@ionic/react-router';
import { Redirect, Route } from 'react-router-dom';
import Menu from './components/Menu';
import Page from './pages/Page';

/* Core CSS required for Ionic components to work properly */
import '@ionic/react/css/core.css';

/* Basic CSS for apps built with Ionic */
import '@ionic/react/css/normalize.css';
import '@ionic/react/css/structure.css';
import '@ionic/react/css/typography.css';

/* Optional CSS utils that can be commented out */
import '@ionic/react/css/padding.css';
import '@ionic/react/css/float-elements.css';
import '@ionic/react/css/text-alignment.css';
import '@ionic/react/css/text-transformation.css';
import '@ionic/react/css/flex-utils.css';
import '@ionic/react/css/display.css';

/* Theme variables */
import './theme/variables.css';

/* This import is for page */
import LatestNews from './pages/latest_news/latest_news';
import TeammatePhotos from './pages/teammate_photos/teammate_photos';

setupIonicReact();

const App: React.FC = () => {
  return (
    <IonApp>
      <IonReactRouter>
        <IonSplitPane contentId="main">
          <Menu />
          <IonRouterOutlet id="main">
            <Route path="/" exact={true}>
              <Redirect to="/latest_news" />
            </Route>
            <Route path="/latest_news" exact={true}>
              <LatestNews />
            </Route>
            <Route path="/teammate_photos" exact={true}>
              <TeammatePhotos />
            </Route>
            <Route path="/folder/:name" exact={true}>
              <Page />
            </Route>
          </IonRouterOutlet>
        </IonSplitPane>
      </IonReactRouter>
    </IonApp>
  );
};

export default App;
\end{lstlisting}

\begin{lstlisting}[language=HTML, caption=Menu.tsx]
import {
  IonContent,
  IonIcon,
  IonItem,
  IonLabel,
  IonList,
  IonListHeader,
  IonMenu,
  IonMenuToggle,
  IonNote,
} from '@ionic/react';

import { useLocation } from 'react-router-dom';
import { archiveOutline, archiveSharp, bookmarkOutline, cameraOutline, cameraSharp, heartOutline, heartSharp, mailOutline, mailSharp, newspaper, newspaperOutline, newspaperSharp, paperPlaneOutline, paperPlaneSharp, trashOutline, trashSharp, warningOutline, warningSharp } from 'ionicons/icons';
import './Menu.css';

interface AppPage {
  url: string;
  iosIcon: string;
  mdIcon: string;
  title: string;
}

const appPages: AppPage[] = [
  {
    title: 'Latest News',
    url: '/latest_news',
    iosIcon: newspaperOutline,
    mdIcon: newspaperSharp
  },
  {
    title: 'Teammate Photos',
    url: '/teammate_photos',
    iosIcon: cameraOutline,
    mdIcon: cameraSharp
  }
];

const Menu: React.FC = () => {
  const location = useLocation();

  return (
    <IonMenu contentId="main" type="overlay">
      <IonContent>
        <IonList id="inbox-list">
          <IonListHeader>Rugby Indonesia</IonListHeader>
          <IonNote></IonNote>
          {appPages.map((appPage, index) => {
            return (
              <IonMenuToggle key={index} autoHide={false}>
                <IonItem className={location.pathname === appPage.url ? 'selected' : ''} routerLink={appPage.url} routerDirection="none" lines="none" detail={false}>
                  <IonIcon aria-hidden="true" slot="start" ios={appPage.iosIcon} md={appPage.mdIcon} />
                  <IonLabel>{appPage.title}</IonLabel>
                </IonItem>
              </IonMenuToggle>
            );
          })}
        </IonList>
      </IonContent>
    </IonMenu>
  );
};

export default Menu;

\end{lstlisting}

\begin{lstlisting}[language=HTML, caption=latest\_news.tsx]
import {
    IonButtons, IonContent, IonHeader, IonMenuButton, IonPage, IonTitle, IonToolbar,
    IonButton, IonCard, IonCardHeader, IonCardTitle, IonCardContent, IonModal, useIonViewWillLeave, useIonViewDidEnter, IonBackButton, useIonViewWillEnter
} from '@ionic/react';
import { useEffect, useState } from 'react';
import './latest_news.css';

import bannerImage from '../../images/sub-header-news.png';
import homeIcon from '../../images/home_icon.png';
import axios from 'axios';
import xml2js from 'xml2js';

const NewsPage: React.FC = () => {
    const [newsItems, setNewsItems] = useState([]);
    const [isOpen, setIsOpen] = useState(false);
    const [selectedItemIndex, setSelectedItemIndex] = useState(-1);

    useEffect(() => {
        const fetchData = async () => {
            try {
                const response = await axios.get('https://dnartworks.rugbyindonesia.or.id/indonesianrugby/news/list.xml');
                const xml = response.data;
                const result = await xml2js.parseStringPromise(xml);
                const data = result.rss.channel[0].item;

                const extractedNews = data.map((item: { description: any[]; title: any[]; 'content:encoded': any[]; }) => {
                    const extractedContent = item.description[0];
                    const regexThumbnail = /src=&quot;(.*?)&quot;/;
                    const matchThumbnail = regexThumbnail.exec(extractedContent);
                    let getThumbnail = '';
                    if (matchThumbnail) {
                        getThumbnail = matchThumbnail[1];
                    }

                    let extractedTitle = item.title[0];

                    const extractedContentEncoded = item['content:encoded'][0];
                    
                    return {
                        thumbnail: getThumbnail,
                        title: extractedTitle,
                        description: extractedContent,
                        content: extractedContentEncoded
                    };
                });
                setNewsItems(extractedNews);
            } catch (error) {
                console.error('Error fetching data:', error);
            }
        };
        fetchData();
    }, []);

    const openModal = (index: number) => {
        setSelectedItemIndex(index);
        setIsOpen(true);
    };

    const closeModal = () => {
        setIsOpen(false);
        setSelectedItemIndex(-1);
    };

    return (
        <IonPage>
            <IonHeader>
                <IonToolbar>
                    <IonButtons slot="start">
                        <a href='/latest_news'>
                            <img src={homeIcon} alt="home-icon" className='home-icon' />
                        </a>
                    </IonButtons>
                    <IonButtons slot="end">
                        <IonMenuButton />
                    </IonButtons>
                    <IonTitle>
                        Persatuan Rugby
                        Union Indonesia
                    </IonTitle>
                </IonToolbar>
            </IonHeader>
            <IonContent className="ion-padding">
                <div className="news-section-image">
                    <img src={bannerImage} alt='latest-news-banner' />
                </div>

                {newsItems.map((item, index) => (
                    <IonCard key={index}>
                        <img alt='latest-news-image' src={item.thumbnail} />
                        <IonCardHeader>
                            <IonCardTitle>{item.title}</IonCardTitle>
                        </IonCardHeader>
                        <IonCardContent>
                            <div dangerouslySetInnerHTML={{ __html: item.description }}></div>
                            <a onClick={() => openModal(index)}>Read More...</a>
                        </IonCardContent>
                    </IonCard>
                ))}
                <IonModal isOpen={isOpen} onDidDismiss={closeModal}>
                    <IonHeader>
                        <IonToolbar>
                            <IonTitle>{selectedItemIndex !== -1 ? newsItems[selectedItemIndex].title : ''}</IonTitle>
                            <IonButtons slot="end">
                                <IonButton onClick={closeModal}>CLOSE</IonButton>
                            </IonButtons>
                        </IonToolbar>
                    </IonHeader>
                    <IonContent className="ion-padding">
                        <img alt='latest-news-image' src={selectedItemIndex !== -1 ? newsItems[selectedItemIndex].thumbnail : ''} />
                        <h1>{selectedItemIndex !== -1 ? newsItems[selectedItemIndex].title : ''}</h1>
                        {selectedItemIndex !== -1 && (
                            <div dangerouslySetInnerHTML={{ __html: newsItems[selectedItemIndex].content }}></div>
                        )}
                    </IonContent>
                </IonModal>
            </IonContent>
        </IonPage>
    );
};

export default NewsPage;
\end{lstlisting}

\begin{lstlisting}[language=HTML, caption=latest\_news.css]
.home-icon{
  width: calc(32rem / 16);
  height: auto;
}

figure img{
  height: 100%;
  width: 100%;
}

ion-toolbar {
  --background: url('../../images/header.png');
  --color: white;
}
\end{lstlisting}

\begin{lstlisting}[language=HTML, caption=teammate\_photos.tsx]
import React, { useEffect, useState } from 'react';
import { IonButtons, IonContent, IonHeader, IonMenuButton, IonPage, IonTitle, IonToolbar, IonButton, IonIcon, IonModal, IonCol } from '@ionic/react';
import { Camera, Photo, CameraResultType, CameraSource } from '@capacitor/camera';

import './teammate_photos.css';

import bannerImage from '../../images/sub-header-photo.png';
import homeIcon from '../../images/home_icon.png';

import { PhotoImages } from "./photoImages";
import PhotoGallery from './photoGallery';
import { appsSharp, camera } from 'ionicons/icons';

import frame1 from '../../images/frame/frame01.png';
import frame2 from '../../images/frame/frame02.png';
import frame3 from '../../images/frame/frame03.png';
import frame4 from '../../images/frame/frame04.png';
import frame5 from '../../images/frame/frame05.png';
import frame6 from '../../images/frame/frame06.png';
import frame7 from '../../images/frame/frame07.png';
import frame8 from '../../images/frame/frame08.png';
import frame9 from '../../images/frame/frame09.png';
import frame10 from '../../images/frame/frame10.png';

const frames = [frame1, frame2, frame3, frame4, frame5, frame6, frame7, frame8, frame9, frame10];

const TeammatePhotosPage: React.FC = () => {
    const [images, setImages] = useState<PhotoImages[]>([]);
    const [isOpen, setIsOpen] = useState(false);
    const [takenPhoto, setTakenPhoto] = useState<Photo | null>(null);

    useEffect(() => {
        const fetchData = async () => {
            try {
                const response = await fetch('https://dnartworks.rugbyindonesia.or.id/indonesianrugby/photos/list.json');
                const data = await response.json();
                const result = data.data;

                if (!Array.isArray(result)) {
                    throw new Error('Data is not in expected format');
                }

                const photoUrls = result.map((item: any) => {
                    const photoUrl = item;
                    const modPhotoUrl = photoUrl.replace("/images/", "/");
                    return modPhotoUrl;
                });
                const photoImages = photoUrls.map((url: string) => ({ webviewPath: url }));
                setImages(photoImages);
            } catch (error) {
                console.error('Error fetching data:', error);
            }
        };

        fetchData();
    }, []);

    const takePicture = async () => {
        const image = await Camera.getPhoto({
            quality: 100,
            allowEditing: false,
            resultType: CameraResultType.Base64,
            source: CameraSource.Camera
        });

        setTakenPhoto(image);
        setIsOpen(true);  // Buka modal setelah foto diambil
    };

    const choosePicture = async () => {
        const image = await Camera.getPhoto({
            quality: 100,
            allowEditing: false,
            resultType: CameraResultType.Base64,
            source: CameraSource.Photos
        });

        setTakenPhoto(image);
        setIsOpen(true);  // Buka modal setelah foto dipilih
    };

    const editPhoto = async (base64data: string, frame: string): Promise<string> => {
        return new Promise((resolve) => {
            const canvas = document.createElement('canvas');
            const ctx = canvas.getContext('2d');
            const img = new Image();
            const frameImg = new Image();

            img.onload = () => {
                canvas.width = 400;
                canvas.height = 400;
                ctx!.drawImage(img, 0, 0, canvas.width, canvas.height);
                frameImg.src = frame;
                frameImg.onload = () => {
                    ctx!.drawImage(frameImg, 0, 0, canvas.width, canvas.height);
                    resolve(canvas.toDataURL('image/jpeg'));
                };
            };

            img.src = `data:image/jpeg;base64,${base64data}`;
        });
    };

    const uploadPhoto = async (base64data: string) => {
        try {
            const userId = "anonymous";

            const response = await fetch('https://dnartworks.rugbyindonesia.or.id/indonesianrugby/photos/upload.json', {
                method: 'POST',
                headers: {
                    'Content-Type': 'application/x-www-form-urlencoded'
                },
                body: `userId=${userId}&photo=${base64data}`
            });

            if (response.ok) {
                console.log('Photo uploaded successfully!');
            } else {
                console.error('Failed to upload photo:', response.statusText);
            }
        } catch (error) {
            console.error('Error uploading photo:', error);
        }
    };

    const handleFrameSelection = async (frame: string) => {
        if (takenPhoto) {
            const editedBase64Data = await editPhoto(takenPhoto.base64String!, frame);
            await uploadPhoto(editedBase64Data);
            window.location.reload();
        }
        setIsOpen(false);
    };

    return (
        <IonPage>
            <IonHeader>
                <IonToolbar>
                    <IonButtons slot="start">
                        <a href='/latest_news'>
                            <img src={homeIcon} alt="home-icon" className='home-icon' />
                        </a>
                    </IonButtons>
                    <IonButtons slot="end">
                        <IonMenuButton />
                    </IonButtons>
                    <IonTitle>
                        Persatuan Rugby
                        Union Indonesia
                    </IonTitle>
                </IonToolbar>
            </IonHeader>
            <IonContent className="ion-padding">
                <div className='teammate-photos-banner'>
                    <img src={bannerImage} alt="Teammate-Photos-Banner" />
                </div>
                <br></br>
                <IonButton color="primary" expand='block' onClick={takePicture}>
                    <IonIcon slot="start" icon={camera}></IonIcon>
                    Take Photo
                </IonButton>

                <IonButton color='primary' expand='block' onClick={choosePicture}>
                    <IonIcon slot="start" icon={appsSharp}></IonIcon>
                    Load from Library
                </IonButton>

                <PhotoGallery photos={images} />

                <IonModal isOpen={isOpen} onDidDismiss={() => setIsOpen(false)}>
                    <IonHeader>
                        <IonToolbar>
                            <IonTitle>Select Frame</IonTitle>
                            <IonButtons slot="end">
                                <IonButton onClick={() => setIsOpen(false)}>CANCEL</IonButton>
                            </IonButtons>
                        </IonToolbar>
                    </IonHeader>
                    <IonContent className="ion-padding">
                        <div className="frame-selection">
                            {frames.map((frame, index) => (
                                <IonCol size = "6" key={index}>
                                    <img
                                        key={index}
                                        src={frame}
                                        alt={`Frame ${index + 1}`}
                                        className="frame-option"
                                        onClick={() => handleFrameSelection(frame)}
                                        width={150}
                                        height={150}
                                    />
                                </IonCol>
                            ))}
                        </div>
                    </IonContent>
                </IonModal>
            </IonContent>
        </IonPage>
    );
};

export default TeammatePhotosPage;
\end{lstlisting}

\begin{lstlisting}[language=HTML, caption=teammate\_photos.css]
.home-icon{
    width: calc(32rem / 16);
    height: auto;
  }

ion-toolbar {
  --background: url('../../images/header.png');
  --color: white;
}
\end{lstlisting}

\begin{lstlisting}[language=HTML, caption=photoGallery.tsx]
import { IonCol, IonGrid, IonImg, IonRow } from "@ionic/react";
import { PhotoImages } from "./photoImages";
import React from "react";

type Props = {
    photos: PhotoImages[],
}

const PhotoGallery: React.FC<Props> =({photos}) => {
    return (
        <IonGrid>
            <IonRow>
                {photos.map((photo, idx) =>(
                    <IonCol size = "6" key={idx}>
                        <IonImg src={photo.webviewPath}/>
                    </IonCol>
                ))}
            </IonRow>
        </IonGrid>
    );
}

export default PhotoGallery;
\end{lstlisting}

\begin{lstlisting}[language=HTML, caption=photoImages.tsx]
export interface PhotoImages
{
    filePath: string;
    webviewPath?: string;
}
\end{lstlisting}

% \lstinputlisting[language=Java, caption=MyCode.java]{./Lampiran/MyCode.java} 

