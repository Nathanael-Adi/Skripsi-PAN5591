%versi 3 (22-07-2020)
\chapter{Landasan Teori}
\label{chap:teori}

Pada bab 2 ini, akan dijelaskan dasar teori terkait dengan aplikasi Rugby Indonesia saat ini, Ionic dan Capacitor. 

\section{Rugby Indonesia App}
\label{sec:skripsi} 
    Aplikasi Rugby Indonesia merupakan aplikasi resmi dari Persatuan Rugby Union Indonesia yang dapat memberikan informasi terbaru mengenai olahraga rugby di Indonesia. Aplikasi ini dapat memberikan notifikasi langsung mengenai berita terakhir, turnamen yang akan datang, dan informasi lainnya. Selain itu, aplikasi ini juga memungkinkan pengguna untuk mengambil gambar dan menunjukkan dukungan mereka dengan rakan-rakan penggemar rugby lainnya. Terdapat juga aplikasi multimedia pengenalan olahraga rugby berbasis Android yang dapat memberikan informasi sejarah, peraturan, dan peralatan serta tempat-latihan rugby di beberapa Kabupaten/Kota. Aplikasi Rugby Indonesia tersedia di Google Play Store dan dapat diunduh secara gratis. Aplikasi ini juga tersedia di GitHub dengan menggunakan framework ionic versi 2 untuk versi iOS.

% Rencananya akan diisi dengan penjelasan umum mengenai buku skripsi.

% \dtext{11-12} 

% \section{\LaTeX}
\section{ReactJS}
\label{sec:latex}

% Mengapa menggunakan \LaTeX{} untuk buku skripsi dan apa keunggulan/kerugiannya bagi mahasiswa dan pembuat template. 

% \dtext{13-14}

ReactJS atau React adalah sebuah library JavaScript yang digunakan untuk membangun user interface yang interaktif. ReactJS berisi kumpulan snippet kode JavaScript yang disebut "komponen" yang bisa digunakan berulang kali untuk mendesain antarmuka pengguna. ReactJS bukanlah framework JavaScript, karena hanya bertugas untuk merender komponen area tampilan aplikasi. ReactJS dapat digunakan untuk membuat aplikasi web dan mobile. Beberapa fitur dan kelebihan ReactJS antara lain:

\begin{itemize}
    \item Reusable Components: Dengan ReactJS, Anda bisa menggunakan lagi komponen yang sudah dikembangkan menjadi aplikasi. Sebab, ReactJS adalah library yang open-source, sehingga Anda bisa membangun komponen siap pakai, yang akan mempercepat proses development aplikasi web kompleks.
    \item Virtual DOM: ReactJS menggunakan Virtual DOM, yang memungkinkan perubahan pada tampilan aplikasi hanya terjadi pada bagian yang berubah saja, tanpa harus merender ulang seluruh tampilan aplikasi. Hal ini membuat aplikasi menjadi lebih cepat dan efisien.
    \item SEO-Friendly: ReactJS bisa memaksimalkan optimisasi mesin pencari (SEO) aplikasi web dengan meningkatkan performanya. Sebab, implementasi Virtual DOM merupakan salah satu faktor yang memengaruhi kecepatan.
    \item Learn Once, Write Anywhere: ReactJS tidak membuat asumsi tentang sisa stack pengguna, sehingga pengguna dapat mengembangkan fitur baru di React tanpa menulis ulang kode yang ada. React juga dapat me-render di server dengan menggunakan Node dan menjalankan aplikasi seluler menggunakan React Native.
    \item UI Interaktif: ReactJS dapat disebut sebagai “Learn One – Write Anywhere” library, karena baik dalam pengembangan aplikasi web dan mobile, React mengikuti pola desain yang sama, memfasilitasi proses transisi. Menggunakan JavaScript polos dan React, Anda dapat membuat UI yang kaya untuk aplikasi asli, serta didukung oleh platform iOS dan Android. 
\end{itemize}

Dengan kelebihan-kelebihan tersebut, ReactJS menjadi pilihan yang masuk akal baik untuk startup maupun perusahaan.

\section{Ionic 7 Framework}
\label{sec:template}
 
% Akan dipaparkan bagaimana menggunakan template ini, termasuk petunjuk singkat membuat referensi, gambar dan tabel.
% Juga hal-hal lain yang belum terpikir sampai saat ini. 
 
% \dtext{15/-16}

Ionic 7 adalah sebuah framework untuk membangun aplikasi mobile hybrid menggunakan HTML5, CSS, dan AngularJS. Framework ini dirilis pada tanggal 29 Maret 2023 dan memiliki beberapa perbaikan yang diusulkan oleh komunitas Ionic. Beberapa fitur baru di Ionic 7 antara lain:

\begin{itemize}
    \item Inline Overlays: Cara yang lebih efisien untuk bekerja dengan kontrol formulir seperti Toggle atau Input. Komponen Item dan Label tidak lagi diperlukan, dan setiap kontrol formulir menangani konten label secara langsung. Perubahan ini mengurangi boilerplate kode dengan menghilangkan persyaratan ion-item dan ion-label.
    \item Performa yang Lebih Baik: Ionic 7 secara signifikan meningkatkan performa Tabs. Pada Ionic React dan Ionic Vue, pengembang dapat mengharapkan peningkatan performa hingga 70\% saat beralih tab. Pengembang Ionic Angular dapat mengharapkan waktu inisialisasi komponen Ionic yang lebih baik berkat optimasi di Stencil.
    \item Kompatibilitas Vite yang Lebih Baik: Ionic 7 menghapus titik masuk Common JS untuk Ionic React dan Ionic Vue untuk membuat setiap paket lebih mudah digunakan dengan Vite dan Vitest.
\end{itemize}

Pengembang dapat mengikuti Panduan Migrasi Ionic 7 untuk memperbarui aplikasi Ionic 6 yang sudah ada. Pada saat tugas akhir ini dibuat, Ionic 7 mendukung Angular 14+, React 17+, dan Vue 3.0.6+.

\subsection{UI Components}
UI Components pada Ionic 7 adalah kumpulan komponen yang digunakan untuk membangun antarmuka pengguna aplikasi mobile hybrid. Komponen-komponen ini memungkinkan pengembang untuk dengan cepat membangun antarmuka pengguna yang menarik dan responsif.

\subsubsection{Action Sheet}
Action Sheet dari Ionic atau yang bisa disebut juga ion-action-sheet merupakan sebuah komponen yang berguna untuk memunculkan dialog. Dialog tersebut akan melakukan pemberhentian sementara terhadap aplikasi yang sedang dijalankannya dan pengguna harus memilih pilihan yang berada di dalam dialog tersebut.

\subsubsection{Accordion}
Accordion berfungsi untuk mengurangi ruang vertikal dalam mengorganisir informasi yang ingin ditampilkan. Accordion memiliki 2 komponen yaitu ion-accordion dan ion-accordion-group. Ketika menggunakan accordion, ion-accordion harus berada di dalam ion-accordion-group.

\subsubsection{Alert}
Alert pada Ionic berfungsi untuk memberikan informasi serta mengumpulkan informasi dari pengguna menggunakan input dari pengguna. Sama seperti Action Sheet, Alert juga biasanya dimunculkan di atas konten dari aplikasi, namun Alert biasanya berada di tengah konten aplikasi, sedangkan Action Sheet muncul berada dari bawah aplikasi.

% \subsection{Tabel}  
% Berikut adalah contoh pembuatan tabel. 
% Penempatan tabel dan gambar secara umum diatur secara otomatis oleh \LaTeX{}, perhatikan contoh di file bab2.tex untuk melihat bagaimana cara memaksa tabel ditempatkan sesuai keinginan kita.

% Perhatikan bawa berbeda dengan penempatan judul gambar gambar, keterangan tabel harus diletakkan di atas tabel!!
% Lihat Tabel~\ref{tab:contoh1} berikut ini:

% \begin{table}[H] %atau h saja untuk "kira kira di sini"
% 	\centering 
% 	\caption{Tabel contoh}
% 	\label{tab:contoh1}
% 	\begin{tabular}{cccc}
% 		\toprule
% 		& $v_{start}$ & $\mathcal{S}_{1}$ & $v_{end}$\\

% 		\midrule
% 		$\tau_{1}$ & 1 & 12& 20\\
% 		$\tau_{2}$ & 1 &  & 20\\
% 		$\tau_{3}$ & 1 & 9 & 20\\
% 		$\tau_{4}$ & 1 &  & 20\\

% 		\bottomrule
		
% 	\end{tabular} 
% \end{table}
% Tabel~\ref{tab:cthwarna1} dan Tabel~\ref{tab:cthwarna2} berikut ini adalah tabel dengan sel yang berwarna dan ada dua tabel yang bersebelahan. 
% \begin{table}[H]
% 	\begin{minipage}[c]{0.49\linewidth}
% 		\centering
% 		\caption{Tabel bewarna(1)}
% 		\label{tab:cthwarna1}
% 		\begin{tabular}{ccccc}
% 			\toprule
% 			 & $v_{start}$ & $\mathcal{S}_{2}$ & $\mathcal{S}_{1}$ & $v_{end}$\\
			
% 			\midrule
% 			$\tau_{1}$ & 1 & 5 \cellcolor{green}& 12& 20\\
% 			$\tau_{2}$ & 1 & 8 \cellcolor{green}& & 20\\
% 			$\tau_{3}$ & 1 & 2/8/17 \cellcolor{green}& 9 & 20\\
% 			$\tau_{4}$ & 1 & \cellcolor{red}& & 20\\
			
% 			\bottomrule

% 		\end{tabular}
% 	\end{minipage}
% 	\begin{minipage}[c]{0.49\linewidth}
		
% 		\centering 
% 		\caption{Tabel bewarna(2)}
% 		\label{tab:cthwarna2}
% 		\begin{tabular}{ccccc}
% 			\toprule
% 			 & $v_{start}$ & $\mathcal{S}_{1}$ & $\mathcal{S}_{2}$ & $v_{end}$\\
			
% 			\midrule
% 			$\tau_{1}$ & 1 & 12& 5 \cellcolor{red} &20\\
% 			$\tau_{2}$ & 1 &  &  8 \cellcolor{green} &20\\
% 			$\tau_{3}$ & 1 & 9 & 2/8/17 \cellcolor{green} &20\\
% 			$\tau_{4}$ & 1 &   & \cellcolor{red} &20\\
			
% 			\bottomrule
		
% 		\end{tabular}
% 	\end{minipage}
% \end{table}

 
% \subsection{Kutipan}
% \label{subs:kutipan} 
% Berikut contoh kutipan dari berbagai sumber, untuk keterangan lebih lengkap, silahkan membaca file referensi.bib yang disediakan juga di template ini.
% Contoh kutipan:
% \begin{itemize}
% 	\item Buku:~\cite{berg:08:compgeom} 
% 	\item Bab dalam buku:~\cite{kreveld:04:GIS}
% 	\item Artikel dari Jurnal:~\cite{buchin:13:median}
% 	\item Artikel dari prosiding seminar/konferensi:~\cite{kreveld:11:median}
% 	\item Skripsi/Thesis/Disertasi:~\cite{lionov:02:animasi}~\cite{wiratma:10:following}~\cite{wiratma:22:later}
% 	\item Technical/Scientific Report:~\cite{kreveld:07:watertight}
% 	\item RFC (Request For Comments):~\cite{RFC1654}
% 	\item Technical Documentation/Technical Manual:~\cite{Z.500}~\cite{unicode:16:stdv9}~\cite{google:16:and7}
% 	\item Paten:~\cite{webb:12:comm}
% 	\item Tidak dipublikasikan:~\cite{wiratma:09:median}~\cite{lionov:11:cpoly}
% 	\item Laman web:~\cite{erickson:03:cgmodel}  
% 	\item Lain-lain:~\cite{agung:12:tango}
% \end{itemize}    
  
% \subsection{Gambar}

% Pada hampir semua editor, penempatan gambar di dalam dokumen \LaTeX{} tidak dapat dilakukan melalui proses {\it drag and drop}.
% Perhatikan contoh pada file bab2.tex untuk melihat bagaimana cara menempatkan gambar.
% Beberapa hal yang harus diperhatikan pada saat menempatkan gambar:
% \begin{itemize}
% 	\item Setiap gambar {\bf harus} diacu di dalam teks (gunakan {\it field} {\sc label})
% 	\item {\it Field} {\sc caption} digunakan untuk teks pengantar pada gambar. Terdapat dua bagian yaitu yang ada di antara tanda $[$ dan $]$ dan yang ada di antara tanda $\{$ dan $\}$. Yang pertama akan muncul di Daftar Gambar, sedangkan yang kedua akan muncul di teks pengantar gambar. Untuk skripsi ini, samakan isi keduanya.
% 	\item Jenis file yang dapat digunakan sebagai gambar cukup banyak, tetapi yang paling populer adalah tipe {\sc png} (lihat Gambar~\ref{fig:ularpng}), tipe {\sc jpg} (Gambar~\ref{fig:ularjpg}) dan tipe {\sc pdf} (Gambar~\ref{fig:ularpdf})
% 	\item Besarnya gambar dapat diatur dengan {\it field} {\sc scale}.
% 	\item Penempatan gambar diatur menggunakan {\it placement specifier} (di antara tanda  $[$ dan $]$ setelah deklarasi gambar.
% 	Yang umum digunakan adalah {\bf H} untuk menempatkan gambar {\bf sesuai} penempatannya di file .tex atau  {\bf h} yang berarti "kira-kira" di sini. \\
% 	Jika tidak menggunakan {\it placement specifier}, \LaTeX{} akan menempatkan gambar secara otomatis untuk menghindari bagian kosong pada dokumen anda.
% 	Walaupun cara ini sangat mudah, hindarkan terjadinya penempatan dua gambar secara berurutan. 	
% 	\begin{itemize}
% 		\item Gambar~\ref{fig:ularpng} ditempatkan di bagian atas halaman, walaupun penempatannya dilakukan setelah penulisan 3 paragraf setelah penjelasan ini.
% 		\item Gambar~\ref{fig:ularjpg} dengan skala 0.5 ditempatkan di antara dua buah paragraf. Perhatikan penulisannya di dalam file bab2.tex!
% 		\item Gambar~\ref{fig:ularpdf} ditempatkan menggunakan {\it specifier} {\bf h}.
% 	\end{itemize}
% \end{itemize}
 
% \dtext{17-18}
% \begin{figure} 
% 	\centering  
% 	\includegraphics[scale=1]{ular-png}  
% 	\caption[Gambar {\it Serpentes} dalam format png]{Gambar {\it Serpentes} dalam format png} 
% 	\label{fig:ularpng} 
% \end{figure} 

% \dtext{19-20}
% \begin{figure}[H]
% 	\centering  
% 	\includegraphics[scale=0.5]{ular-jpg}  
% 	\caption[Ular kecil]{Ular kecil} 
% 	\label{fig:ularjpg} 
% \end{figure} 
% \dtext{21-22}

% \begin{figure}[ht] 
% 	\centering  
% 	\includegraphics[scale=1]{ular-pdf}  
% 	\caption[ {\it Serpentes} betina]{ {\it Serpentes} jantan} 
% 	\label{fig:ularpdf} 
% \end{figure} 
 
% \subsection{Kode Program}

% Kode program dalam bahasa tertentu seringkali harus ditulis di dalam bab, bukan hanya dilampirkan di bagian Lampiran. 
% Kode~\ref{kode:aneh} menampilkan penggunaan karakter-karakter yang umum digunakan dalam sebuah program yang ditulis dengan bahasa C.


% \begin{lstlisting}[language=Java, caption=Kode untuk menampilkan karakter-karakter aneh, label=kode:aneh]
% // This does not make algorithmic sense, 
% // but it shows off significant programming characters.

% #include<stdio.h>

% void myFunction( int input, float* output ) {
% 	switch ( array[i] ) {
% 		case 1: // This is silly code
% 			if ( a >= 0 || b <= 3 && c != x )
% 				*output += 0.005 + 20050;
% 			char = 'g';
% 			b = 2^n + ~right_size - leftSize * MAX_SIZE;
% 			c = (--aaa + &daa) / (bbb++ - ccc % 2 );
% 			strcpy(a,"hello $@?"); 
% 	}
% 	count = ~mask | 0x00FF00AA;
% }

% // Fonts for Displaying Program Code in LATEX
% // Adrian P. Robson, nepsweb.co.uk
% // 8 October 2012
% // http://nepsweb.co.uk/docs/progfonts.pdf

% \end{lstlisting}

% \subsection{Notasi}

% Simbol-simbol (matematika) yang sering digunakan sepanjang penulisan skripsi, dapat dimasukkan ke dalam ``Daftar Notasi''. Daftar ini ada di halaman depan sebelum Bab~\ref{chap:intro}.
% Cara memasukkan sebuah simbol ke dalam Daftar Notasi adalah menggunakan perintah \verb|\nomenclature|. Contoh:
% \begin{center}
%     \verb|\nomenclature[]{$A$}{luas kandang ular}|    
% \end{center}
% \nomenclature[]{$A$}{luas kandang ular}
% \nomenclature[]{$n$}{banyaknya ular}
% \nomenclature[]{$k$}{jumlah kepala per seekor ular\nomrefpage}
% Argumen opsional digunakan untuk mengurutkan notasi. Silahkan lihat sendiri dokumentasi package \verb|nomencl|

